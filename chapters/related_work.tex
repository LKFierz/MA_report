\chapter{Related Work}
\label{sec:relwork}

% TODO: Split this into three parts, one being general Rotorcraft landing, one being specific autonomy frameworks and one being stereo

\section{Autonomous UAV Landing}

Autonomous safe landing is perhaps the most important part of a rotorcraft's mission. It comes therefore as no surprise, that tremendous amounts of work have been accomplished in the pursuit of achieving this crucial feature. 

Camera sensors are highly advantageous for navigation due to their lightweight nature and the extensive research dedicated to their development over the years. The minimal weight of these sensors makes them particularly suitable for applications where payload capacity is a critical concern, such as in the case of a rotorcraft Mars missions. Decades of intensive research have culminated in highly sophisticated algorithms and methodologies that leverage the rich data captured by visual sensors and enable the daunting task of autonomous landing.

\citep{Saripalli2002VisionBasedLanding,Falanga2017QuadrotorLanding} and \citep{Mu2023VisionBasedLanding} use artificial landing markers as indications of valid landing sites. While \citep{Saripalli2002VisionBasedLanding} and \citep{Mu2023VisionBasedLanding} use stationary markings, \citep{Falanga2017QuadrotorLanding} enabled a rotorcraft to land on a moving target. These approaches,  though useful in urban environments,  are not applicable on uncharted terrain as found on Mars.

\citep{Bosch2006AutonomousDetection,Brockers2011AutonomousLanding,Desaraju2015VisionBased} and \citep{Brockers2014TowardsAutonomous} pursue implementations based on homography assumptions. This is not possible in our setup as we cannot assume homographic conditions on Mars' rough terrain. %TODO What exactly did they do?

A very handy tool for the creation of depth maps to segment landing sites on are range sensors like Lidar as \citep{Trawny2015FlightTesting, Luna2017Evaluation, Johnson2002LidarBased} and \citep{Scherer2012AutonomousLanding} show. As for our purposes a rotorcraft has to fly on Mars' 1\% air density however, weight is a limiting constraint rendering Lidar sensor a suboptimal choice.

For the Mars Mission's lander NASA has used a vision based strategy using a predefined map of Mars' surface and a downwards facing monocular camera to orient the lander in the predefined map\citep{Johnson2020Mars2020}. When compared to a lander, however, rotorcrafts need to consider much smaller hazards. At 25 cm/pixel for images and 1 m/pixel for the DEM, the available Mars footage from the HiRISE camera on the Mars Reconnaissance Orbiter are not sufficient in resolution to supply prior information to the landing process of a UAV. Rover images could be used as well as Ingenuity's footage however the usage of this data would limit possible flight areas significantly.

\citep{Johnson2005VisionGuided} uses a similar approach as the one used by LORNA. Compared to LORNA's Landing Site Detection \citep{LSD1,LSD2} however, a non-robot-centric DEM is used. The advantage of LORNA's approach is the implicit drift handling by considering the robot-centered terrain map.

Modern approaches like \citep{Fankhauser2014RobotCentric, Forster2015Continuous} and \citep{Daftry2018Robust} use a 2.5D terrain representation similar to the setup used in this project.

Other novel approaches use learning based methods as did \citep{Neves2024Multimodal, Abdollahzadeh2022SafeLandingZones} and \citep{TovanchePicon2024RealTimeSafeValidation}. Though certainly promising regarding accuracy and in the long run definitely a pathway to consider, learning based methods come with significant costs in the context of the task at hand. First of all considering the limitations present in Mars missions, the probable additional computational overhead from learning based methods can not be neglected. Furthermore, neural network based solutions give up simplicity and interpretability for the benefit of precision. This is not to be underestimated in a hostile environment such as Mars' rough terrain where perfect accuracy is required. Additionally, learning based methods require substantial training data which, in the context of autonomous UAV landing, is not available in large quantities. Especially when the training data needs to come from another planet. Lastly, missions flown on Mars pose their unique challenges compared to conventional flights. Such specific knowledge can be leveraged in traditional methods to tailor an efficient procedure to the needs at hand. An example of this could be the knowledge about Mars' terrain monotony and the often present lack of texture.

\section{Autonomy Architectures}

\section{Stereo Methods}\label{sec:rel_stereo}