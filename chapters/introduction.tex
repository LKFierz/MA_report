\chapter{Introduction}
\label{sec:introduction}

With the Ingenuity rotorcraft's lifecycle coming to an end, the question about future mars rotorcrafts and their capabilities draws ever closer.

For the future NASA develops two different rotorcraft Mars concepts. The first is a rotorcraft for low altitudes equipped with a mechanical gripper. It is envisioned to be an alternative way of transporting Mars samples to the retrieval station should the Perseverance rover fail to do so.

Secondly, for future large distance missions, NASA is conceptualizing a Mars Science Helicopter (MSH) project. The aspirations for such a rotorcraft are on one hand to cover farther distances at high altitudes with accurate state estimation and on the other to land safely, autonomously and reliably in previously unknown terrain. These two feats allow a helicopter to perform much advanced science missions compared to Ingenuity. 

The LOng Range NAvigation (LORNA) project that I have been involved with is working on a concept to tackle the second project's challenges while dealing with the constraints that rotorcraft missions on Mars provide us with. These are namely a limitation on the size and weight of the drone, a constraint on compuational power due to the deployment on limited embedded processors and lastly a large delay in communication which makes adaptive remote control from Earth impossible.

\section{Objective}

The conclusive high level objective of the LORNA science concept is the achievement of long range safe navigation including global localization, safe landing site detection and full system autonomy. The navigation endeavour is tackled using a visual-inertial odometry state estimator which uses map based localization to achieve global localization as well. Landing site acquisition is achieved using structure from motion based 3D reconstruction and a multi resolution depth elevation map used for landing zone segmentation. Finally, a state machine-based autonomous framework orchestrates the entire procedural workflow.

The endeavour in this thesis was to create a front to back landing mechanism that combines the existing vision based landing site detection algorithm with the autonomous framework. In order to accomplish this, both the landing site detection algorithm as well as the autonomy had to be altered. Last but not least given that the structure from motion depth generation depends on lateral movement, which is less desirable for a drone navigating at low altitudes in unfamiliar surroundings, the utilization of a stereo camera for low altitude 3D reconstruction presents a viable solution to attain real-time depth perception without necessitating lateral displacement.

\section{My Contribution}
In this work, I established the interface between the vision based landing site detection algorithm and the autonomous framework in order to make informed landing decisions based on detected landing sites. A safe and efficient landing mechanism was implemented in the existing autonomy. This mechanism utilizes a novel stereo camera 3D reconstruction procedure to avoid lateral motion at low altitudes.

\begin{itemize}
    \item \textbf{Stereo Camera Depth Alternative}

    A stereo camera was implemented in the simulated drone model in order to get stereo sensor images. Additionally, a stereo camera depth node was put in place as an alternative to SFM to supply the landing site detection algorithm with a point cloud at low altitudes without the need for lateral motion. 

    An automatic switch was inserted between the SFM node and the stereo camera depth node by utilizing the laser range finder sensor on board. This allows for minimal computational overhead as only one depth creation node runs at a time.
    \item
\end{itemize}
% \subsection{Stereo Camera Depth Alternative}
% A stereo camera was implemented in the simulated drone model in order to get stereo sensor images. Additionally, a stereo camera depth node was put in place as an alternative to SFM to supply the landing site detection algorithm with a point cloud at low altitudes without the need for lateral motion. 

% An automatic switch was inserted between the SFM node and the stereo camera depth node by utilizing the laser range finder sensor on board. This allows for minimal computational overhead as only one depth creation node runs at a time.

\subsection{Flight Analysis at 100m Altitude}
The visual landing site detection pipeline has not been tested at 100m altitudes prior to this work. As LORNA plans to fly at this cruise altitude, the quality of the structure from motion point clouds as well as the LSD performance is mission critical and was therefore evaluated. %TODO how much do I really analyze it?
\subsection{Autonomy LSD Interface and Landing Site Handling}
The landing site detection output initially only consisted of the location of a landing site. This output was enhanced to consider many more characteristics in order for the autonomy to make an informed decision with regards to what spot to select.

The autonomy was enhanced to correctly receive the incoming landing sites and process them in an efficient way that
\subsection{Behavior Tree for Adaptive Decisions}

Using the existing behavior tree framework from the autonomy, an adaptive landing behavior was implemented. For this, the alteration of numerous existing actions as well as the addition of new action modules interacting with the landing sites was necessary.
\subsection{Simulation Setup}
As just recently the switch was made to Gazebo Garden the entire visual pipeline (SFM + LSD) had never run with this simulation environment before. Therefore I implemented the changes necessary to run the landing site detection procedure on the Gazebo sensor input. Additionally whilst implementing the stereo camera and attempting to put in place a simulated depth camera for ground truth it became apparent that the Gazebo depth camera implementation is incorrect neglecting the set intrinsic parameters. Altering Gazebo's source code the implementation could be fixed.
\subsection{Deployment of LSD Pipeline onto an Embedded Processor}
Currently the used processor is modalAI's voxl2. Both the structure from motion as well as the landing site detection software did not run out of the box having an incompatible dependency handling with the voxl's AARCH architecture. Resolving these issues I was able to run the landing site detection pipeline with the structure from motion depth supply on the voxl2 using a collected rosbag of images and IMU poses.

\section{Organisation of this Thesis}
\subsection{Related Work}
As is custom, I will introduce the reader to what has been done in this area. Main focus will be placed on vision-based landing site detection procedures and previous work on autonomous landing.
\subsection{System Overview}
The entire project overview will be introduced. Emphasis lies on the methods that I have heavily interacted with in this thesis. These are mainly the structure from motion depth generation, the landing site detection mechanism and the autonomous framework.
\subsection{Methodology}
Here I conceptually lay out the high level structure of the implemented work in this thesis. The two key contributions are introduced and their subtasks listed.
\subsection{Stereo Camera Depth Alternative}
This chapter introduces the stereo camera depth part of the methodology. I will go into the reasoning why a stereo camera is necessary as a low altitude depth alternative. Additionally, I analyze the stereo option theoretically and describe the process of implementing it in the existing project structure. Lastly it is qualitatively compared to a depth camera based ground truth.
\subsection{Autonomous Landing Procedure}
Here I will lay out the core contribution of this project which combines the existing system with the novel contributions of this work in order to put together a front to back automous landing procedure. First I will describe the conceptual landing behavior and later on I explain the practical implementation. Lastly the working pipeline is shown in a case example of a science mission flown in simulation.
\subsection{Evaluation}
Here I introduce the test setup according to which I performed repeated randomized simulation flights. I introduce the outcome defining metrics and the results of the test flights. Lastly these results are analyzed conceptually and with specific examples. %TODO Is this still the case? Did I go for specific examples?
\subsection{Conclusion}
I summarize the novel contributions of this work and conclusively assess the characteristics and quality of the final landing pipeline. Shortcomings of the approach are pointed out and remedies are discussed.
\subsection{Outlook}
Further enhancements of the current systems are layed out and alternatives for future iterations are discussed. Also emphasis is placed on current insufficiencies and the necessity of resolving them.