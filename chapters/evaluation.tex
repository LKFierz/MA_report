\chapter{Evaluation}
\label{chapter:evaluation}

\section{A Performance Evaluation of the SFM-LSD Pipeline at 100m Altitude}
\section{Experimental Setup}
\subsection{Test Maps}
In the following all the performed experiments were flown on the following two maps:
\begin{itemize}
    \item Arroyo Map - Prerecorded Gazebo map from the Arroyo Seco area outside the East entrance of the Jet Propulsion Laboratory.
% TODO Image of Arroyo
    \item Rough Test Map - A synthetically created map using Blender, designed not to have any safe landing site throughout its entire area. 
% TODO Image of rough test map
\end{itemize}
\subsection{Drone Spawn}
The drone was either spawned repetetively from a default location on the ground (The start location from when the actual fields tests were performed) or from a random location. For a simplicity way of  avoiding terrain collisions, the drone was spawned on a randomly positioned disk at 40m altitude. The start disk's size was only 0.5m in diameter which prevented it from being considered too good of a landing site by LSD. This is important because the platform implicitly gains quality due to the fact, that it is located higher up than the terrain, leading to a lower distance to the drone when flying at mission altitude.


\section{Testing}


