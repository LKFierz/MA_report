\chapter*{Evaluation}
\addcontentsline{toc}{chapter}{Evaluation}

Here I split the stereo system's application into three scenarios:
\begin{itemize}
    \item Low altitude hover
    \item Low altitude slow lateral movement
    \item Low altitude vertical ascent / descent
\end{itemize}

So self-evidently the stereo is going to fly at low altitudes. To get the exact boundaries of the performance one can simply look at existing stereo algorithm performance metrics:

\section{Max Drone Height for Stereo View}
As is widely known in literature a depth value can be derived from a stereo disparity result using the following formula:

\begin{equation}\label{eq:1.0}
    z = \frac{f \cdot b}{d}
\end{equation}

Where z is the depth of a perceived point, d the measured disparity, f the focal length in pixels and b the baseline of the two stereo cameras.

Taking the derivative of z w.r.t. d we get

\begin{equation}
    \frac{\partial z}{\partial d} = - \frac{f  \cdot b}{z^2}
\end{equation}

And substituting (\ref{eq:1.0}) into d we get for the depth error:

\begin{equation}
    {\partial z} = \frac{z^2}{f  \cdot b}\partial d
\end{equation}

Leaving away the sign as for our application there lies equal danger in a point being perceived too close and too far away.

And for the maximum height:

\begin{equation}
    z_{\text{max}} = \sqrt{\frac{\Delta z_{\text{max}} \cdot b \cdot f}{\Delta d}}
\end{equation}
\begin{equation}
    \Delta z = \frac{z^2 \cdot \Delta d}{b \cdot f}
    \label{eq:depth_error}
\end{equation}

Assuming a standard stereo camera and considering the setup which was initially installed on the modalAI Sentinel drone that I worked with we have a focal length of 256 pixels and a baseline of 9.2cm.

Thus, when assuming a disparity error of 0.25 pixels and allowing a maximum rock or dent size of 10cm we get for the maximum altitude:

\begin{equation}
    z_{\text{max}} = \sqrt{\frac{0.1m \cdot 0.092 m \cdot 256}{0.25}} = 3.069m
\end{equation}

So with the stereo setup with this baseline a maximum altitude of 3.069 meters can be reached while still detecting mission threatening roughness. As described above this matches the desired application of the stereo setup.

% Show plot different depth errors for different heights
% Show images of practice with lsd



\section{Baseline Analysis}
The baseline faces two main constraints:
\begin{itemize}
    \item The baseline has to be large enough for the drone to detect ground roughness of mission threatening size at a safe minimal altitude.
    \item Yet it also needs be small enough to get a decent stereo image when hovering just above the ground.
\end{itemize}

\subsection{Baseline Minimum}

Using the same formula as above and rearranging it for the baseline we get:

\begin{equation}
    b = \frac{z^2}{f \cdot \Delta z} \Delta d
\end{equation}

So the minimum baseline curve looks like this:

\subsection{Baseline Maximum}

Looking at the stereo setup as two pinhole cameras the setup looks like this:

% Make image using matcha.io

This leads to the following formula for the fov overlap:


